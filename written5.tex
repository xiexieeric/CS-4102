\documentclass[a4paper,12pt]{article}
\title{CS 4102 Written HW5 - Miscellaneous}
\author{Eric Xie}
\usepackage{listings}
\begin{document}
\maketitle
\noindent \textbf{1.} \\
The probability of the pivot being in the middle 50th percentile is 50\%, since there is a 25\% chance that the pivot is below the 25th percentile and 100-75=25\% chance that the pviot is above the 75th percentile. \\
Since the base case does not select a pivot, we only need to find the probablity of this happening on the non-leaf nodes of the recursion tree. \\
Probability = $0.5^n$ , where  n = \# of internal nodes \\\\

If this were guaranteed, the worst case is that the pivot falls at the edge of the middle 50\%, dividing the array into n/4 and 3n/4 values. \\
Since it takes linear time to perform pivots on each portion, the recurrence can be expressed as $T(n)=T(n/4)+T(3n/4)+n$\\
Using the substitution method, we assume that the runtime is $O(nlogn)$\\
That means there exists c and $n_0$ such that $T(n_0)\leq cnlog(n_0)$
This is true for $n_0=2$, $T(2)\leq c*2*1$, when $c\geq2$ \\

\noindent For the inductive step:\\
$T(k)\leq cklog(k)$\\
$= T(k/4)+T(3k/4)+k \leq cklogk$\\
$= cklog(k/4)/4+3cklog(3k/4))/4+k \leq cklogk$\\
$= c(\frac{k}{4}log(\frac{k}{4})+\frac{3k}{4}log(\frac{3k}{4}))+k \leq cklogk$\\
$= \frac{ck}{4}(log(\frac{k}{4})+3log(\frac{3k}{4}))+k \leq cklogk$\\
$= \frac{c}{4}(log(\frac{k}{4})+3log(\frac{3k}{4}))+1 \leq clogk$\\
$= \frac{c}{4}(log(k)-log(4)+3log(3k)-3log(4)+1 \leq clogk$\\
$= \frac{c}{4}(log(k)-2+3log(3)+3log(k)-6)+1 \leq clogk$\\
$= \frac{c}{4}(4log(k)+3log(3)-8)+1 \leq clogk$\\
$= clog(k)+\frac{3clog(3)}{4}-2c+1 \leq clogk$\\
$= \frac{3clog(3)}{4}-2c \leq -1$\\
$= 3clog(3)-8c \leq -4$ \\
$c<0.57$ \\
No contradiction, so $T(n) \in O(nlogn)$\\



\noindent \textbf{2.} \\
At the last minute of any sequence, we can always fire the gun without consequence, since there is no point in charging the gun any further. \\
This will kill up to C(j) zombies, making the total kill count arr[currentTime-j]. \\
Since we want to maximize total kills, we need to choose j so that C(j)+arr[currentTime-j] is maximized, checking every prior time.  \\
Using the above properties, we can solve the problem using the following algorithm:
\begin{lstlisting}
n=len[Z]
Initialize an array of size n to store the total zombies killed so far.
for i in (0,n):
   if(i==0):
	return max(C(0), Z[0]);
   else:
      temp=arr[0];
      for j in (1,i):
         temp = max(temp, C[j]+arr[i-k]
      arr[i]=temp;
\end{lstlisting}
If we fire the gun, we can kill C[k] of the current zombies, with k being the last time the gun was fired. 
If we don't fire the gun, we are assuming that the  
\noindent \textbf{3.} \\
First, we must sort the list of lines, L, by increasing slope. \\
When the lines are sorted by slope, the first and last lines in the list are always visible. \\
In addition, when looking at a line, $L_{n-2}$, the last line, $L_n$, and any line in between, $L_{n-1}$ \\
Let $I_i$ be the point at which $L_{n-2}$ and $L_{n-1}$ intersect \\
Let $I_j$ be the point at which $L_{n-2}$ and $L_n$ intersect \\
If the $I_i$ is left of $I_J$, then $L_n$ cannot fully dominate $L_{n-1}$, and all three lines are visible. \\
If the $I_i$ is right of $I_J$, then $L_n$ completely dominates $L_{n-1}$, and only $L_{n-2}$ and $L_n$ are visible. \\

Using the properties above, we can simply add subsequent lines to our list of visible lines, removing $L_m$ as necessary to give the list of visible lines up to 
$L_n$.
\begin{lstlisting}
visible = []
if(len(L)==1)
   visible.append(L[0])
else
   visible.append(L[0])
   visible.append(L[1])
for(int i=2;i<len(L);i++):
   n=len(visible);
   while(intersect(visible[n-2],visible[n-1]).x  < intersect(visible[n-2], L[i] ).x  && n>1):
      visible.removeLast()
      n = n-1
   visible.append(L[i])
\end{lstlisting}
The runtime of this algorithm is O(n). At worst, the while loop has to remove every $I_{n-1}$, but once a line is removed it is never considered again, so the worst number of removals is n-2. The number of insertions is always n, so together the runtime is 2n-2, or O(n)
\end{document}
\documentclass[a4paper,12pt]{article}
\title{CS 4102 Written HW2 - Greedy}
\author{Eric Xie}
\usepackage{listings}
\begin{document}
\maketitle
\noindent 1. Master Theorem:\\
$T(n)=2T\left(\frac{n}{2}\right)+n^2$
a=2, b=2, k=2 \\
$a<b^k$, thus this is case 1 of the Master Theorem: $T(n)\in\Theta(n^2)$\\\\
Substitution Method: \\
Assume $T(n)\in\Theta(n^2)$, thus $T(n)\leq cn^2$ \\
Base Case: $n_0=2$\\
$T(2)=2T(1)+4<4c$\\
$2+4<4c$\\
$6<4c$\\
$c>\frac{3}{2}$\\\\
IH: $T(k-1)\leq c(k-1)^2$\\
IS: $T(k-1+1)\leq ck^2$\\
$2T(k/2)+k^2 \leq ck^2$\\
$2c(k/2)^2+k^2 \leq ck^2$\\
$2c\frac{k^2}{4}+k^2\leq ck^2$\\
$\frac{2c}{4}+1\leq c$\\
$2c+4\leq 4c$\\
$4\leq 2c$\\
$2\leq c$, which agrees with the assumption that $c\geq3/2$\\
Thus $T(n)\in\Theta(n^2)$
\\\\
Directly Solve:\\
$T(n)=2T\left(\frac{n}{2}\right)+n^2$\\
$=2[2T(n/4)+(n/2)^2]+n^2$ \\
$=4T(n/4)+\frac{n^2}{2}+n^2$\\
$=2^kT(\frac{n}{2^k})+\frac{(2^k-1)n^2}{2^{k-1}}$\\\\
$T(1)=1$\\
$\frac{n}{2^k}=1$\\
$n=2^k$\\
$k=log_2{n}$\\\\\\

$T(n)=2^{\log_2n}T(\frac{n}{2^{\log_2n}})+\frac{(2^{\log_2n}-1)n^2}{2^{\log_2n-1}}$\\
$=nT(\frac{n}{n})+\frac{(n-1)n^2}{\frac{n}{2}}$\\
$=n+2(n^2-n)$\\
$=n+2n^2-2n$\\
$=2n^2-n$\\
$=\Theta(n^2)$\\

2. \\
1) Base case: n=2. Do f(first index,second index) and return the index which is greater. Otherwise...\\
2) If n is even, do f(first half, second half), and recurse on the half that is greater\\
3) If n is odd, omit the median value, and do f(first half, second half)\\
	If the result is +1 or -1, recurse on the side that is greater\\
	Otherwise if the result is 0, return the median index\\
$T(n)=T(n/2)+n/2$\\
According to Master Theorem, a=1, b=2, k=1\\
Since $a<b^k$, this is case 1, which means the runtime is $\Theta(n)$\\\\
3. 
\begin{lstlisting}
public int median(int h1, int h2, int s1, int s2){
	int n1=h2-h1+1;
	int n2=s2-s1+1;
	int k1=ceiling((h1+h2)/2)
	int k2=ceiling((s1+s2)/2)
	int m1=happy.query(k1);
	int m2=sad.query(k2);
	if((n1==2 && n2<=2) || (n1<=2 && n2==2)
		arr=happy.subArray(h1,h2).append(sad.subArray(s1,s2))
		return arr[arr.size-2]	
	if(n1==1 && n2==1)
		return max(m1,m2);
	else:
		if(m1>m2)
			return median(h1,k1,k2,s2)
		if(m1<m2)
			return median(k1,h2,s1,k2)
} 
\end{lstlisting}
4. \\
T(n)=2T(n/2)+2n\\
By the Master Theorem, a=2, b=2, k=1\\
Since $a<b^k$, this is case 1 of the Master Theorem, meaning $T(n)\in\Theta(nlog(n))$\\\\
5. \\
Proof by induction:\\
Base Case: n=1, the algorithm selects the higher of the salaries, which is correct\\
IH: For n=k, the algorithm does not eliminate the true median\\
IS: For n=k+1, the algorithm does not eliminate the true median\\
F.S.C, assume that the algorithm eliminates the 1/4 of the salaries containing the median\\
Let the two medians be m1 and m2 such that m1<m2\\\\
Case 1: The median is contained in the eliminated interval [1,m1]\\
The intervals [m1,n] and [m2,n], which by definition contains half of the total salaries, are both greater the "median", but there is another interval [m1,m2] above the median, making more than half the salaries greater than the median, which contradicts the definition of median.\\\\
Case 2: The median is contained in the elminiated interval [m2,n]\\
The intervals [1,m1] and [1,m2], which by definition contains half of the total salaries, are both less the "median", but there is another interval [m1,m2] below the median, making more than half the salaries less than the median, which contradicts the definition of median.\\

Therefore, the algorithm does not eliminate the median for n=k+1 if it does not eliminate the median for n=k. Therefore the algorithm never removes the median while continuing to reduce n to n/2. \\


\newpage
6. 
\begin{lstlisting} 
int index1=0
int counter1=0
int index2=0
for(int i=0;i<n;i++){
	if(counter1==0){
		index2=index1;
		index1=i
		counter1++;
	}
	else{
		if(equivalent(index,i))
			counter1++;
		else
			counter1--;
	}
}
counter1=0;
int counter2=0
for(int i=0;i<n;i++){
	if(equivalent(index,i))
		counter1++;
	if(equivalent(index2,i))
		counter2++;
}
if(counter1>=(n/2) || counter2>=(n/2))
	return true;
else
	return false;
\end{lstlisting}
\end{document}